%===========================================================================
% ============= Setup and formatting of the document ========================
% ===========================================================================
\documentclass[a4paper,10pt]{article}
% ===========================================================================
% ======================== Load packages ====================================
% ===========================================================================


% Add mathematical symbols, useful if you use a lot of formulas. This package automatically loads other important mathematical packages.
\usepackage{amsmath}
\usepackage{float}
% The titletoc option of the appendix package produces a title "Appendix" when creating the appendix section with \appendix
\usepackage[titletoc]{appendix}
\usepackage{float}
\usepackage{placeins}
% Extends the array and tabular environments
\usepackage{array}

% Defines the language you want to use for the autocorrect (may not exist in all languages)
%\usepackage[english]{babel}		%English
\usepackage[english]{babel}	%German							

% Allows to write bold math letters (to show vectors without the arrow)
\usepackage{bm}

% Enhances the quality of tables, another package for tables
\usepackage{booktabs}

% Specify that all captions should be written with a smaller font size than the main text
\usepackage{caption} \captionsetup{font=small} % smaller caption font size
% Personalise captions:
\captionsetup[figure]{labelfont={bf},name={Fig.},labelsep=period}
\captionsetup[table]{labelfont={bf},name={Table},labelsep=period}

% To write text in color. In the text for instance write something in black and \textcolor{red}{something in red}
\usepackage{color}

% Color package for tables, to allow e.g. cells to be colored
\usepackage{colortbl}

% Another table package, for alignment
\usepackage{ctable}

% Greatly improves the flexibility of the list environmants itemize, enumerate and description
\usepackage{enumitem}

% Improves the management of the floats (i.e. the "free" things floating in the text like figures or tables). With this package you can use the order \begin{figure}[H] to force the figure to appear in the text at the location where you put it. Otherwise the figures will be placed close to that location but somewhere where it looks "nice", which is not always satisfying.
\usepackage{float}

% Easy use of German umlauts 
\usepackage[T1]{fontenc}
\usepackage[utf8]{inputenc} 

\usepackage{listings}
\usepackage{listings}
\usepackage{xcolor}

\lstdefinestyle{mystyle}{
    backgroundcolor=\color{white},   
    commentstyle=\color[rgb]{0,0.5,0}, % Green comments
    keywordstyle=\color[rgb]{0,0.5,0}, % Green keywords
    numberstyle=\tiny\color[rgb]{0.6,0.6,0.6}, % Light gray line numbers
    stringstyle=\color[rgb]{0.8,0.5,0.8}, % Light purple strings
    basicstyle=\ttfamily\color{black}\footnotesize,
    breakatwhitespace=false,             
    breaklines=true,                     
    captionpos=b,                        
    keepspaces=true,                     
    numbers=left,                        
    numbersep=10pt,  % Increased space between line numbers and code                      
    showspaces=false,                    
    showstringspaces=false,              
    showtabs=false,                      
    tabsize=2                            
}
\lstset{style=mystyle}


% Provides generic symbols for math and text
\usepackage{gensymb}

\usepackage{csquotes}

% Defines the geometry of the page, the margins
\usepackage[left=2.5cm,right=2.5cm,top=2.5cm,bottom=2.5cm]{geometry}

% Allows to include figures from e.g. jpg file
\usepackage{graphicx}

% So that one can click on references. Set reference color.
%\usepackage[colorlinks,bookmarks=false,linkcolor=blue,urlcolor=blue,color=blue]{hyperref}
\usepackage{hyperref}

% To add filler text
\usepackage{lipsum}

% Reference and citation package, to work with author-date and numerical citations
% The option comma separates several citations with commas.
% The option numbers would use the nnumerical citation style (otherwise if not defined uses author-date)
%\usepackage{apacite}
%\usepackage[sort]{natbib}
\usepackage[
  style=apa,
  backend=biber
]{biblatex}

\addbibresource{reserachmodule_agp_bib.bib}


% Allow a manual modification of paragaph indentation
% Sets the length of the indentation
\usepackage{parskip}
\setlength{\parindent}{1cm}

% Set spacing between lines. Examples: \singlespacing or \doublespacing or \setstretch{1.2}
% The definition of the spacings differs between different LaTeX packages or MS Word. If you want to get confused about the spacing, have a look there:
% https://tex.stackexchange.com/questions/65849/confusion-onehalfspacing-vs-spacing-vs-word-vs-the-world/65985
% Here we will go for \setstretch{1.5}
\usepackage{setspace} \setstretch{1.25}

% Allows the figure caption to be on the side (default=outer side)
% Note: does not work with [H]
\usepackage{sidecap}

% For proper use of units
\usepackage{siunitx}

% To allow subcaptions, e.g. Figure 1 a).... b)...
\usepackage{subcaption}

% To change the font add the corresponding package, e.g below. Fonts with serif should be used for written reports while fonts without serif should be used for presentations.
%\usepackage{times}

% To modify title setup and referencing
% tocbibind adds non-numbered sections to the table of contents
% The nottoc does not show the table of contents in the table of contents
\usepackage{titlesec}
\usepackage[nottoc]{tocbibind}

% Allows footnotes below tables, but still within the table environment.
\usepackage{threeparttable}

% Allows special types of underlining.
% The normalem option keeps emphasized words in italic but not underlined (important for a correct bibliography)
\usepackage[normalem]{ulem} % underline possible without underlinig emph. words

% More symbols
\usepackage{wasysym}

% Allows the text to wrap around figures
\usepackage{wrapfig}

% For fun, using xkcd colors for text or table cells. Now you can decently hand in reports with text color "pea soup" or "light yellowish green", and there are other even more inventive names...
% https://xkcd.com/color/rgb/
\usepackage{xkcdcolors}


% Automatic referencing for tables, figures, equations or sections
% This package automatically detects the type of object you are referring to. Details in Section 6
\usepackage{cleveref} % must be loaded after hyperref package
%\crefname{object}{your name for singular}{and for plural}
\crefname{table}{Table}{Tables}
\crefname{section}{Section}{Sections}
\crefname{figure}{Fig.}{Figs.}
\crefname{equation}{Eq.}{Eqs.}
\Crefname{figure}{Figure}{Figures}

% Tables
\usepackage{multicol}
\usepackage{booktabs} % For formal tables
\usepackage{multirow} % Required for multirow cells
\usepackage{array}    % Required for centering in the multirow cells
\usepackage{lipsum} % Zum Generieren von Blindtext

% Rotate figures
\usepackage{rotating}

% Reset figure and table counting in appendices
\usepackage{chngcntr}

% Rotate page to landscape for large figures
\usepackage{pdflscape}
% ===========================================================================
% ====================== Misc. commands =====================================
% ===========================================================================

% Define lines for tables
\newcommand{\otoprule}{\midrule[\heavyrulewidth]}
\newcommand*\rot{\rotatebox{90}} % rotate table entry

% Define equation numbering to be dependent on the chapters. good for long reports, to avoid having equation 1 to 200, instead in Chapter 1 have equation 1.1 to 1.30, in Chapter 2 equations 2.1 to 2.20, etc.
%\numberwithin{equation}{section}

% To avoid single lines at the top and bottom of pages at all costs (give an infinite weight (=9999))
\widowpenalty=9999
\clubpenalty=9999

% Define the width of subcaptions
\captionsetup[subfigure]{width=0.9\textwidth}

% Fancy headers
\usepackage{fancyhdr}
\pagestyle{fancy}
\fancyhead{} % clear all header fields
\fancyfoot{} % clear all footer fields
\fancyhead[L]{Researchmodule in Applied Geophysics}
\fancyhead[R]{WS 2025}
\fancyfoot[C]{\thepage}
% Placement in header: L left, C centre, R right

% for blank pages before chapters without header
\makeatletter
\def\cleardoublepage{\clearpage\if@twoside \ifodd\c@page\else
\hbox{}
\vspace*{\fill}
\thispagestyle{empty}
\newpage
\if@twocolumn\hbox{}\newpage\fi\fi\fi}
\makeatother

\newcommand\nobrkhyph{\mbox{-}}

\begin{document}
\pagenumbering{gobble} % no page numbering on the title page
\thispagestyle{empty} % disable fancy headers on the title page

\begin{center} % center the text only for the title page
	\vspace*{200pt}
	\textbf{\huge{Comparison of three source types in Snow-Air and Earth-Snow-Air models in Salvus ??????}}\\
	\vspace{75pt}
	\textbf{\Large{Salome Bachmann (473462)}} \\\large{salome.bachmann@rwth-aachen.de}\\
	\vspace{7pt}
	\vspace{65pt}
	\rule{\linewidth}{.5pt}
		\large{\textbf{Research Module in Applied Geophysics}}\\
        \large{\textbf{WS 2025}}\\
		\large{\textbf{RWTH Aachen}}\\
		\vspace{7pt}
		\large{\today }\\
	\rule{\linewidth}{.5pt}
	\vspace{45pt}
\end{center}

\newpage
\begin{abstract}
a short abstract (max. 300 words)  
\end{abstract}

\newpage
\tableofcontents

\newpage
\pagenumbering{arabic}\setcounter{page}{1}

\section{Introduction}
%Introduction with a definition of the scientific question
\noindent Snow avalanches are snow masses that descend slopes \parencite{schweizerSnowAvalancheFormation2003}. Snow avalanches can occur in different types, such as snow slab avalanches, where a slab of snow slides \parencite{gaumeSnowFractureRelation2017}. Such avalanches are among the most prominent hazards in mountainous regions \parencite{gaumeSnowFractureRelation2017}, they can be triggered by rapid loading near the surface, for example by a person skiing over them, or by a dynamite source \parencite{schweizerSnowAvalancheFormation2003}. The cracks caused by this propagate and eventually lead to material failure , the release of the slab is caused by an increase of shear stress in a weak layer in the snowpack \parencite{habermannInfluenceSnowpackLayering2008}. \\

\noindent Understanding the signatures of different types of sources is crucial for understanding crack propagation. This research project aims to compare simulated wave propagation of three different source types (pure acoustic pressure source, elastic vector source,and a moment tensor source in 2d) in two different models, which are earth-snow-air and snow-air. The aim of this comparison is to determine which models and source types are useful to understand crack propagation in snow leading to slab avalanches, which will be analyzed in the Master Thesis. These considerations are not only based on the results of the different types of simulations, but also computation time and memory. 

\section{Methodology}
\subsection{Introduction to Mondaic Salvus Software}
\noindent Before simulations were created, the Salvus software was installed according to the Mondaic website. Next, the Salvus tutorials concerning setup of seismic events, creation of different types of models, setup of simulations as well as output of the wavefield were worked through. This helped a lot with the setup of the simulations which is described in the following section.

\subsection{Setup of Simulations - Pure Acoustic \& Elastic Case}
\noindent The first simulation which was set up is set in a homogenous domain 1m by 1m. The density (\rho) as well as p- and s-wave velocities have values of snow, which were taken from \cite{guillemotEffectSnowfallChanges2021}, table 2. All the medium parameters which were used in the three different models can be seen in table \ref{tab:material_properties}. The source, which is located halfway at the top of the domain (x=0.5, y=1) is a scalar point 2D source with force in negative y-direction. The field which was recorded in this case is the first time derivative of \phi, meaning the acoustic veloicty. This simulation was created to verify the setup of the domain and get more familiar with simulation setup. The s-waves were not recorded in this simulation, this is to minimize simulation time. This simulation will not be analyzed as a result. \\

\begin{table}[H]
\centering
\caption{Material Parameters (density and p and s-wave velocities used in the models)}
\label{tab:material_properties}
\begin{tabular}{|l|l|l|l|}
\hline
Material & \begin{tabular}[c]{@{}l@{}}\textbackslash{}rho\\ {[}kg/$\ m^3$|\end{tabular} & \begin{tabular}[c]{@{}l@{}}Vp\\ {[}m/s{]}\end{tabular} & \begin{tabular}[c]{@{}l@{}}Vs\\ {[}m/s{]}\end{tabular} \\ \hline
Snow     & 180                                                                                       & 300                                                    & 150                                                    \\ \hline
Air      & 1.2250                                                                                    & 332                                                    & 0                                                      \\ \hline
Soil     & 2000                                                                                      & 2200                                                   & 880                                                    \\ \hline
\end{tabular}
\end{table}

\noindent Next, a simulation with a snow-layer of 1m thickness and an 1m thick air layer was created, both layers have an x-extent of 30m. This was done by creating numpy-arrays with the edges of the layers in x and the y direction, so for the snow layer this would be [0,30] in x and [0,1] in y. Then, the model parameters were also defined as arrays. Then, a mesh was generated by interpolation between the edge points and the material parameters were assigned to the appropriate layers. Furthermore, absorbing boundaries were added to the bottom of the snow layer, the top of the air layer and all sides. This was done so that there are no reflections from the model boundaries, which is closer to a real-life scenario. First, a pure acoustic simulation with a scalar point 2D source was run, the source was located halfway in the snow layer, the simulation recorded the acoustic velocity field. The parameters used for simulation can be seen in table \ref{tab:simulation_params}. Next, the simulation was run for the elastic case with two different source types: a vector point source with fx=-1 and a moment tensor source with mxy=3e4. The vector source represents an explosive source in the snow, whereas the moment tensor source represents a slip on a surface. The mxy component of the force was chosen assuming a 70kg skier on the snow. The normal force was calculated according to $\ F = m*g = 70 * 9.81 = 700N$, the frictional shear force between the skis and the snow was assumed to be 100N, the slip distance 0.1m and the slip area on two skis 0.3$\ m^3$, the shear modulus \mu of snow $10^6$ Pa. This leads to $ M_0 = \mu * A * D = 10^6 * 0.3 * 0.1 = 3*10^4 N$. \\


\noindent Lastly, a three layered model was set up, which consists of a 1m soil layer, a 1m snow layer and an 1m air layer, all with 30m lateral extent, was set up as described above. Again, absorbing boundaries were added, instead of the bottom of the snow layer the absorbing boundary conditions were added to the bottom of the soil layer.  The medium properties are shown in table \ref{tab:material_properties}.  Again, first the pure acoustic case was considered, with the same source type and field as described above. Then, the elastic case was recorded with a sources of the same types as above located halfaway in the snow layer. The simulation parameters used can be seen in table \ref{tab:simulation_params}. 

\begin{table}[H]
\centering
\caption{Simulation parameters used}
\label{tab:simulation_params}
\begin{tabular}{|l|l|}
\hline
Parameter Name          & Value \\ \hline
Source Center Frequency & 700Hz \\ \hline
Event End Time          & 2s    \\ \hline
Sampling Interval       & 200   \\ \hline
\end{tabular}
\end{table}


\subsection{Physical Considerations}
\noindent To set up the simulation parameters in table \ref{tab:simulation_params} and the different models, a few physical considerations were needed. \\

\noindent Firstly, the extent of the layers needs to be plausible. Not only do avalanches usually occur over large spatial extent, the lateral extension of the model also needs to be large enough so that far-field effects of the waves can be observed. A function in was written in python which performs these calculations.  This was determined by taking the maximum and minimum speed present in the model, the speeds are the same as in table \ref{tab:material_properties}. The source wavelet, which was a ricker wavelet in all cases, was fast-fourier transformed (fft). From the frequency values of the wavelet, the minimum and maximum frequency were determined. The smallest and largest wavelengths to be resolved were determined accoding to $\ \lambda_{min} = \frac{v_{min}}{f_{max}}$ and $\ \lambda_{max} = \frac{v_{max}}{f_{min}}$, where \lambda is the minimum or maximum wavelength, v is the velocity and f is the frequency determined by the fft. To observe far-field effects, the medium needs to be much larger in the x-direction than the maximum wavelength to be resolved. In this case, the maximum wavelength determined was about 2cm, which means that an x-extent of 30m is sufficient.The vertical thickness of the layers was determined by real-world considerations, meaning that they were determined by what is plausible, a 40m snow layer for example would not be very likely. \\

\noindent The center frequency of the source was determined by the layer thicknesses. The maximum wavelength to reslove a layer is about 4 times the thickness of the layer: $\  \lambda = 4*y_{layer}$. From this, wavelength was converted to frequency according to $\ f = \frac{v}{\lambda} $. The velocity which was taken is the maximum velocity in present in the model. With the medium parameters in table \ref{tab:material_properties}, the calculation is $\ \lambda = 4*1 = 4 $, which leads to $\ f = \frac{2200}{4} = 550 Hz$ Based on the layer thicknesses and speeds present, a center frequency of 700Hz was chosen to make sure all layers are resolved well. \\

\noindent The simulation sampling interval is a tradeoff between simulation time, memory needed and resolution. The larger the sampling interval, the fewer samples in time. The appropriate time step was determined by the Nyquist sampling criterion where there need to be at least two sampling points per period: $\ f_{sampling} = \frac{1}{frequency} * \frac{1}{2}$, where the frequency is the maximum meaningful frequency of the wavelet, which can be obtained by plotting the wavelet. 


\section{Results}
\subsection{Pure Acoustic}
\label{results_pure_ac}

\begin{figure}[H]
    \centering
    \begin{subfigure}[b]{0.3\linewidth}
        \centering
        \includegraphics[width=\linewidth]{figures/pure acoustic/snow-air/t=3.png}
        \caption{Timestep 3}
        \label{fig:ac_snowair_t3}
    \end{subfigure}
    \hspace{0.02\linewidth}
    \begin{subfigure}[b]{0.3\linewidth}
        \centering
        \includegraphics[width=\linewidth]{figures/pure acoustic/snow-air/t=6.png}
        \caption{Timestep 6}
        \label{fig:ac_snowair_t6}
    \end{subfigure}

    \vspace{0.5em}

    \begin{subfigure}[b]{0.3\linewidth}
        \centering
        \includegraphics[width=\linewidth]{figures/pure acoustic/snow-air/t=42.png}
        \caption{Timestep 42}
        \label{fig:ac_snowair_t42}
    \end{subfigure}
    \hspace{0.02\linewidth}
    \begin{subfigure}[b]{0.3\linewidth}
        \centering
        \includegraphics[width=\linewidth]{figures/pure acoustic/snow-air/t=61.png}
        \caption{Timestep 61}
        \label{fig:ac_snowair_t61}
    \end{subfigure}

    \caption{Pure Acoustic snow-air model velocity outputs at selected timesteps.}
    \label{fig:ac_snowair_all}
\end{figure}

\noindent Figure \ref{fig:ac_snowair_all} shows the pure acoustic case of the snow-air model. As shown in \ref{fig:ac_snowair_all}, the acoustic waves travel in the snow layer only. Figure \ref{fig:ac_layered_all} shows the pure acoustic case of the earth-snow-air model, where earth extends from 3m to 2m, snow from 2m to 1m depth and air from 1m to 0m, here, the acoustic waves from the scalar point source also only travel in the snow and earth layers. Figure \ref{fig:ac_snowair_all} shows wavefronts over several meters, whereas figure \ref{fig:ac_layered_all} shows much smaller wavefronts. At timestep 3 in the snow-air model, shown in figure \ref{fig:ac_snowair_t3}, the wavefield has already propagated, while at the same same timestep for the earth-snow-air model, shown in figure \ref{fig:ac_layered_t3}, the propagation is much less far. 

\begin{figure}[H]
    \centering
    \begin{subfigure}[b]{0.3\linewidth}
        \centering
        \includegraphics[width=\linewidth]{figures/pure acoustic/layered/t=3.png}
        \caption{Timestep 3}
        \label{fig:ac_layered_t3}
    \end{subfigure}
    \hspace{0.02\linewidth}
    \begin{subfigure}[b]{0.3\linewidth}
        \centering
        \includegraphics[width=\linewidth]{figures/pure acoustic/layered/t=6.png}
        \caption{Timestep 6}
        \label{fig:ac_layered_t6}
    \end{subfigure}

    \vspace{0.5em}

    \begin{subfigure}[b]{0.3\linewidth}
        \centering
        \includegraphics[width=\linewidth]{figures/pure acoustic/layered/t=42.png}
        \caption{Timestep 42}
        \label{fig:ac_layered_t42}
    \end{subfigure}
    \hspace{0.02\linewidth}
    \begin{subfigure}[b]{0.3\linewidth}
        \centering
        \includegraphics[width=\linewidth]{figures/pure acoustic/layered/t=56.png}
        \caption{Timestep 56}
        \label{fig:ac_layered_t56}
    \end{subfigure}

    \caption{Pure Acoustic layered model velocity outputs at selected timesteps.}
    \label{fig:ac_layered_all}
\end{figure}

\noindent Slower propagation of the acoustic wavefield for the three-layer model can also be seen in \ref{fig:ac_layered_t42} and \ref{fig:ac_layered_t56}, it takes the wavefield 0.05273 seconds to reach the end of the model domain, while in the two-layer case the wavefield has almost reached the edge of the domain at 0.02551s, shown in \ref{fig:ac_snowair_t6}. 


\subsection{Elastic}

\subsubsection{Vector Source}
\label{el_vector}

\begin{figure}[H]
    \centering
    \begin{subfigure}[b]{0.3\linewidth}
        \centering
        \includegraphics[width=\linewidth]{figures/elastic/snow-air/vector2d/t=3.png}
        \caption{Timestep 3}
        \label{fig:el_snowair_v2d_t3}
    \end{subfigure}
    \hspace{0.02\linewidth}
    \begin{subfigure}[b]{0.3\linewidth}
        \centering
        \includegraphics[width=\linewidth]{figures/elastic/snow-air/vector2d/t=6.png}
        \caption{Timestep 6}
        \label{fig:el_snowair_v2d_t6}
    \end{subfigure}

    \vspace{0.5em}

    \begin{subfigure}[b]{0.3\linewidth}
        \centering
        \includegraphics[width=\linewidth]{figures/elastic/snow-air/vector2d/t=42.png}
        \caption{Timestep 42}
        \label{fig:el_snowair_v2d_t42}
    \end{subfigure}
   \hspace{0.02\linewidth}
    \begin{subfigure}[b]{0.3\linewidth}
        \centering
        \includegraphics[width=\linewidth]{figures/elastic/snow-air/vector2d/t=61.png}
        \caption{Timestep 61}
        \label{fig:el_snowair_v2d_t61}
    \end{subfigure}

    \caption{Elastic snow-air model outputs with 2D vector source at selected timesteps.}
    \label{fig:el_snowair_v2d_all}
\end{figure}

\begin{figure}[H]
    \centering
    \begin{subfigure}[b]{0.3\linewidth}
        \centering
        \includegraphics[width=\linewidth]{figures/elastic/layered/vector2D/t=3.png}
        \caption{Timestep 3}
        \label{fig:el_layered_v2d_t3}
    \end{subfigure}
    \hspace{0.02\linewidth}
    \begin{subfigure}[b]{0.3\linewidth}
        \centering
        \includegraphics[width=\linewidth]{figures/elastic/layered/vector2D/t=6.png}
        \caption{Timestep 6}
        \label{fig:el_layered_v2d_t6}
    \end{subfigure}

    \vspace{0.5em}

    \begin{subfigure}[b]{0.3\linewidth}
        \centering
        \includegraphics[width=\linewidth]{figures/elastic/layered/vector2D/t=42.png}
        \caption{Timestep 42}
        \label{fig:el_layered_v2d_t42}
    \end{subfigure}
   \hspace{0.02\linewidth}
    \begin{subfigure}[b]{0.3\linewidth}
        \centering
        \includegraphics[width=\linewidth]{figures/elastic/layered/vector2D/t=56.png}
        \caption{Timestep 56}
        \label{fig:el_layered_v2d_t56}
    \end{subfigure}

    \caption{Elastic layered model outputs with 2D vector source at selected timesteps.}
    \label{fig:el_layered_v2d_all}
\end{figure}

\subsubsection{Moment Tensor Source}
\label{el_tensor}

\begin{figure}[H]
    \centering
    \begin{subfigure}[b]{0.3\linewidth}
        \centering
        \includegraphics[width=\linewidth]{figures/elastic/snow-air/momenttensor/t=3.png}
        \caption{Timestep 3}
        \label{fig:el_snowair_mom_t3}
    \end{subfigure}
    \hspace{0.02\linewidth}
    \begin{subfigure}[b]{0.3\linewidth}
        \centering
        \includegraphics[width=\linewidth]{figures/elastic/snow-air/momenttensor/t=6.png}
        \caption{Timestep 6}
        \label{fig:el_snowair_mom_t6}
    \end{subfigure}

    \vspace{0.5em}

    \begin{subfigure}[b]{0.3\linewidth}
        \centering
        \includegraphics[width=\linewidth]{figures/elastic/snow-air/momenttensor/t=42.png}
        \caption{Timestep 42}
        \label{fig:el_snowair_mom_t42}
    \end{subfigure}
   \hspace{0.02\linewidth}
    \begin{subfigure}[b]{0.3\linewidth}
        \centering
        \includegraphics[width=\linewidth]{figures/elastic/snow-air/momenttensor/t=61.png}
        \caption{Timestep 61}
        \label{fig:el_snowair_mom_t61}
    \end{subfigure}

    \caption{Elastic snow-air model outputs with 2D moment tensor source at selected timesteps.}
    \label{fig:el_snowair_mom_all}
\end{figure}



\begin{figure}[H]
    \centering
    \begin{subfigure}[b]{0.3\linewidth}
        \centering
        \includegraphics[width=\linewidth]{figures/elastic/layered/momenttensor2D/t=3.png}
        \caption{Timestep 3}
        \label{fig:el_layered_mom_t3}
    \end{subfigure}
    \hspace{0.02\linewidth}
    \begin{subfigure}[b]{0.3\linewidth}
        \centering
        \includegraphics[width=\linewidth]{figures/elastic/layered/momenttensor2D/t=6.png}
        \caption{Timestep 6}
        \label{fig:el_layered_mom_t6}
    \end{subfigure}

    \vspace{0.5em}

    \begin{subfigure}[b]{0.3\linewidth}
        \centering
        \includegraphics[width=\linewidth]{figures/elastic/layered/momenttensor2D/t=42.png}
        \caption{Timestep 42}
        \label{fig:el_layered_mom_t42}
    \end{subfigure}
   \hspace{0.02\linewidth}
    \begin{subfigure}[b]{0.3\linewidth}
        \centering
        \includegraphics[width=\linewidth]{figures/elastic/layered/momenttensor2D/t=56.png}
        \caption{Timestep 56}
        \label{fig:el_layered_mom_t56}
    \end{subfigure}

    \caption{Elastic layered model outputs with 2D moment tensor source at selected timesteps.}
    \label{fig:el_layered_mom_all}
\end{figure}



\section{Discussion \& Outlook}
\subsection{Discussion}

\subsection{Outlook to Master Thesis Project}
\noindent This research project serves as an useful preparation for the following master thesis. This project was crucial in learning how to use Mondaic Salvus modelling software, which will also be used in the thesis. Furthermore, the models generated and the results from different source types serve as a basis for the thesis project. From this preparation, I now understand for example that, as shown by the results of the different models and source types, including the air layer is not strictly necessary because there is no significant wave propagation in that layer, this will help to optimize model computation time in the thesis. The models used in this research project are much simpler than the ones which will be used in the master thesis, for example there is no topography included in these models and the snow layer is homogenous, which means that new models will need to be made for the thesis in order to represent the conditions on the mountain more accurately. 

\newpage
\section{Bibliography} 
\typeout{}
\printbibliography

\section{Data Management NOT SURE IF THIS IS NECCESSARY}




\end{document}